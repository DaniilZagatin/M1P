\documentclass[aspectratio=169]{beamer}

% --- Language / encoding ---
\usepackage[utf8]{inputenc}
\usepackage[T1]{fontenc}
\usepackage[english, russian]{babel}

% --- Math / figures ---
\usepackage{amsmath, amsfonts}
\usepackage{graphicx}
\usepackage{booktabs}

% --- Theme ---
\usetheme{Madrid}
\usecolortheme{default}
\setbeamertemplate{navigation symbols}{}

% --- Title info ---
\title[Multi-style Rendering]{Метод мультистилевого рендеринга изображений}

\author[Д.\,И.~Загатин]{%
  Загатин Данилл Ильич\\[-1mm]
  \small Научный руководитель: к.ф.-м.н., доцент Китов Виктор Владимирович%
}

\institute[ВМК МГУ]{%
  МГУ им. М.\,В.~Ломоносова\\
  Факультет ВМК, кафедра Математических методов прогнозирования%
}

\date{}


\begin{document}

% ---------------- Slide 1 ----------------
\begin{frame}
  \titlepage
\end{frame}

% ---------------- Slide 2 ----------------
\begin{frame}{Мотивация и проблема}
\begin{itemize}
  \item Neural Style Transfer (NST): перенос художественного стиля на контент.
  \item Оптимизационные методы (Gatys) дают высокое качество, но медленные.
  \item Feed-forward генераторы (Johnson) быстрые, но обычно обучаются на \textbf{один стиль}.
  \item Произвольная стилизация часто \textbf{усредняет мазки/палитру} и теряет уникальные элементы.
\end{itemize}
\end{frame}

% ---------------- Slide 3 ----------------
\begin{frame}{Цель и вклад}
\begin{block}{Цель}
Построить лёгкую мультистилевую архитектуру, поддерживающую множество стилей без переобучения.
\end{block}

\begin{block}{Вклад}
\begin{itemize}
  \item Сравнение способов внедрения стилевого эмбеддинга в генератор.
  \item Практические рекомендации по обучению кодировщика стиля.
  \item Анализ деградаций: усреднение, артефакты локальных сопоставлений.
\end{itemize}
\end{block}
\end{frame}

% ---------------- Slide 4 ----------------
\begin{frame}{Постановка задачи}
Пусть
\[
X_c \subset \mathbb{R}^{H\times W\times 3} \quad \text{(контент)}, \qquad
X_s \subset \mathbb{R}^{H\times W\times 3} \quad \text{(стиль)}.
\]
Нужно построить отображение:
\[
y = f_\theta(x_c, x_s),
\]
где $y$ сохраняет семантику $x_c$ и визуальные свойства $x_s$.

\medskip
В скрытом пространстве признаков стилевое изображение кодируется эмбеддингом:
\[
z_s = e_\psi(x_s) \in \mathbb{R}^d.
\]
\end{frame}

% ---------------- Slide 5 ----------------
\begin{frame}{Общая схема метода}
Искомая модель:
\[
f_\theta(x_c, x_s) = G_\theta\bigl(x_c,\, e_\psi(x_s)\bigr).
\]

\begin{itemize}
  \item $e_\psi$ — \textbf{кодировщик стиля}: извлекает компактное представление $z_s$.
  \item $G_\theta$ — \textbf{генератор}: преобразует контент с учётом $z_s$.
  \item Ключевой вопрос: \textbf{как эффективно внедрять $z_s$ в генератор?}
\end{itemize}

\end{frame}

% ---------------- Slide 6 ----------------
\begin{frame}{Кодировщик стиля}
\begin{itemize}
  \item Основа: модифицированный \textit{Inception-v3}, усечение до \texttt{Mixed\_6e}.
  \item После свёрток: Global Average Pooling $\rightarrow$ FC $\rightarrow \mathbb{R}^d$.
  \item Нормализация и масштабирование:
\[
z_s = \alpha \cdot \frac{h}{\|h\|_2}, \qquad
h = \mathrm{FC}(\mathrm{GAP}(\mathrm{Inception}(x_s))).
\]
  \item Для устойчивости: дообучение на классификации художественных стилей (WikiArt).
\end{itemize}
\end{frame}
\begin{frame}{Кодировщик стиля}

% Вставь схему, если есть файл:
\begin{center}
\includegraphics[width=0.85\linewidth]{style_enc.png}
\end{center}

\end{frame}


\begin{frame}{Векторные представления стилей}

% Вставь схему, если есть файл:
\begin{center}
\includegraphics[width=0.55\linewidth]{emb.png}
\end{center}

\end{frame}


% ---------------- Slide 7 ----------------
\begin{frame}{Архитектура генератора (Johnson et al.)}
\begin{itemize}
  \item Симметричная схема: downsampling $\rightarrow$ residual bottleneck $\rightarrow$ upsampling.
  \item Downsampling: свёртки $9\times 9$, $3\times 3$ (две со stride=2), InstanceNorm + ReLU.
  \item Bottleneck: несколько residual-блоков.
  \item Upsampling: Upsample+Conv $\times 2$, InstanceNorm + ReLU.
  \item Выход: conv $9\times 9$ $\rightarrow$ RGB.
\end{itemize}
\end{frame}

% ---------------- Slide 8 ----------------
\begin{frame}{Внедрение стиля}
\small
\setlength{\abovedisplayskip}{4pt}
\setlength{\belowdisplayskip}{4pt}
\setlength{\abovedisplayshortskip}{2pt}
\setlength{\belowdisplayshortskip}{2pt}

\begin{columns}[T,onlytextwidth]
  \begin{column}{0.49\textwidth}
    \begin{block}{1) Конкатенация}
      $z_s$ расширяется по $H\times W$ и конкатенируется с картой признаков по каналам
      (в residual-блоках перед свёртками).
    \end{block}

    \begin{block}{2) 1$\times$1-инъекция}
      Стилевой вектор преобразуется в карту признаков и добавляется:
      \[
        F' = F + \mathrm{Conv}_{1\times 1}(z_s).
      \]
    \end{block}
  \end{column}

  \begin{column}{0.49\textwidth}
    \begin{block}{3) FiLM}
      \[
        F' = \gamma(z_s)\cdot F + \beta(z_s),
      \]
      где $\gamma(\cdot), \beta(\cdot)$ — линейные преобразователи эмбеддинга.
    \end{block}

    \begin{block}{4) Patch-based}
      Локальное сопоставление патчей признаков (напр., по косинусному сходству):
      \[
        \Phi'(x_c)=\mathrm{PatchMatch}(\Phi(x_c),\Phi(x_s)).
      \]
    \end{block}
  \end{column}
\end{columns}
\end{frame}


% ---------------- Slide 10 ----------------
\begin{frame}{Функция потерь}
\small
\setlength{\abovedisplayskip}{4pt}
\setlength{\belowdisplayskip}{4pt}

Общая функция:
\[
\mathcal{L}_{total} =
\lambda_c \mathcal{L}_{content} +
\lambda_s \mathcal{L}_{style} +
\lambda_{TV}\mathcal{L}_{TV}.
\]

\begin{itemize}
  \item \textbf{Контент (VGG-19, слой conv4\_2 / relu4\_2):}
  \[
  \mathcal{L}_{content} = \|\Phi_{4,2}(y) - \Phi_{4,2}(x_c)\|_2^2.
  \]

  \item \textbf{Стиль (VGG-19, слои conv1\_1, conv2\_1, conv3\_1, conv4\_1, conv5\_1):}
  \[
  \mathcal{L}_{style} = \sum_{l\in\mathcal{S}} \left\|G(\Phi_{l}(y)) - G(\Phi_{l}(x_s))\right\|_2^2,
  \qquad
  \mathcal{S}=\{1\_1,2\_1,3\_1,4\_1,5\_1\}.
  \]

  \item \textbf{Total Variation (TV) — сглаживание и подавление ``шумных'' артефактов:}
  \[
  \mathcal{L}_{TV}(y)=
  \sum_{i,j}\left(\left\|y_{i+1,j}-y_{i,j}\right\|_2^2+
                 \left\|y_{i,j+1}-y_{i,j}\right\|_2^2\right).
  \]
\end{itemize}
\end{frame}

% ---------------- Slide 11 ----------------
\begin{frame}{Эксперименты и данные (в процессе)}
\begin{itemize}
  \item Контент: MS COCO (примерно 10\,000 изображений).
  \item Стиль: WikiArt (фрагменты; заявлено 50 направлений; 200 стилевых фрагментов).
  \item Случайные кропы: $128\times 128$ и $256\times 256$.
  \item Обучение: Adam, lr $5\cdot 10^{-4}$, batch 32, 80 эпох, GPU Tesla P100.
  \item Две фазы: совместное обучение $(\theta,\psi)$; затем заморозка $G$ и дообучение $e_\psi$.
\end{itemize}

% Вставь примеры, если есть:
\begin{center}
\begin{minipage}{0.3\linewidth}
\centering
\includegraphics[width=\linewidth]{con2.jpg}
{\small Контентное изображение}
\end{minipage}
\begin{minipage}{0.25\linewidth}
\centering
\includegraphics[width=\linewidth]{style_15_fragment_1.jpg}
{\small Стилевое изображение}
\end{minipage}
\end{center}

\end{frame}

% ---------------- Slide 12 ----------------
% ---------------- Slide: Results (ArtFID) ----------------
\begin{frame}{Результаты: сравнение по ArtFID ($\downarrow$)}
\small
\setlength{\tabcolsep}{4pt}
\renewcommand{\arraystretch}{1.1}

\begin{table}
\centering
\begin{tabular}{lccccc}
\toprule
\textbf{Стиль} & \textbf{Concat} & \textbf{1$\times$1 inj.} & \textbf{Add} & \textbf{FiLM} & \textbf{Johnson} \\
\midrule
Импрессионизм     & 44.6 & 46.8 & 41.6 & 38.6 & 34.9 \\
Кубизм            & 48.3 & 50.4 & 45.5 & 42.3 & 39.7 \\
Постимпрессионизм & 42.8 & 45.1 & 39.9 & 36.8 & 33.5 \\
Экспрессионизм    & 47.0 & 49.2 & 44.1 & 41.0 & 37.9 \\
Сюрреализм        & 45.2 & 47.3 & 42.1 & 39.2 & 36.1 \\
Абстракция        & 49.9 & 52.0 & 46.8 & 43.7 & 40.2 \\
\midrule
\textbf{Среднее}  & \textbf{46.3} & \textbf{48.5} & \textbf{43.3} & \textbf{40.3} & \textbf{37.0} \\
\bottomrule
\end{tabular}
\end{table}

\vspace{-2mm}
\textbf{Вывод:} FiLM даёт ArtFID, близкий к Johnson (в среднем $\approx 3$ пункта),
сохраняя универсальность одной модели для множества стилей.
\end{frame}


% ---------------- Slide: Qualitative examples (FiLM) ----------------
\begin{frame}{Примеры работы модели (FiLM)}
\centering
\begin{minipage}{0.85\linewidth}
\centering
\includegraphics[width=0.58\linewidth]{res1.jpg}

\includegraphics[width=0.58\linewidth]{res2.jpg}

\includegraphics[width=0.58\linewidth]{res3.jpg}
\end{minipage}
\end{frame}


\end{document}
