\documentclass{article}
\usepackage{arxiv}

\usepackage[utf8]{inputenc}
\usepackage[english, russian]{babel}
\usepackage[T1]{fontenc}
\usepackage{url}
\usepackage{booktabs}
\usepackage{amsfonts}
\usepackage{nicefrac}
\usepackage{microtype}
\usepackage{lipsum}
\usepackage{graphicx}
\usepackage{natbib}
\usepackage{doi}

\title{Метод мультистилевого рендеринга изображений}

\author{
  Загатин Данилл Ильич \\
  МГУ им. М.В. Ломоносова \\
  Факультет ВМК, кафедра Математических методов прогнозирования \\
  Москва, Россия \\
  \texttt{Почта тут} \\
  \And
  Научный руководитель: к.ф.-м.н., доцент Китов Виктор Владимирович \\
  МГУ им. М.В. Ломоносова
}
\date{}

\renewcommand{\shorttitle}{Multi-style Rendering Method}


\hypersetup{
  pdftitle={Метод мультистилевого рендеринга изображений},
  pdfsubject={cs.CV},
  pdfauthor={Загатин Данилл Ильич, Китов Виктор Владимирович},
  pdfkeywords={перенос стиля, произвольная стилизация, стилевой эмбеддинг, AdaIN, FiLM, patch-based},
}

\begin{document}
\maketitle

\begin{abstract}
В работе рассматривается задача произвольной стилизации изображений с использованием сверточных нейронных сетей (CNN). Большинство моделей обучаются для конкретного стиля и успешно передают палитры и текстуры, однако часто не воспроизводят уникальные стилистические особенности, склоняясь к усреднённым мазкам и цветам. На основе методов Гатиса, Джонсона и Гиаси исследуются модификации генератора с кодировщиком стиля и различными способами внедрения эмбеддингов: конкатенацией, 1×1-инъекцией, FiLM-блоками и патчевыми методами. Работа выявляет архитектурные ограничения и предлагает улучшения: усиление тренировки кодировщика, раздельное внедрение компонентов стиля и использование альтернативных функций потерь. Разработана мультистилевая лёгкая архитектура с высоким качеством стилизации и гибкостью применения к разнообразным стилям.
\end{abstract}

\keywords{перенос стиля \and свёрточные нейронные сети \and мультистилевая стилизация \and стилевой эмбеддинг \and FiLM}

\section{Введение}
Перенос художественного стиля (Neural Style Transfer, NST) позволяет преобразовывать изображение, сохраняя его семантическое содержимое и воспроизводя художественные особенности другого изображения. Практическая ценность NST растёт в цифровом искусстве, дизайне, AR/VR и мобильных приложениях, где важны интерактивность, персонализация и низкая стоимость генерации контента.

Классические оптимизационные подходы обеспечивают высокое качество, но требуют длительной итеративной подгонки под каждую новую пару изображений. Быстрые генеративные сети позволяют работать в реальном времени, однако в базовом варианте обучаются под один конкретный стиль, что ограничивает масштабирование. Универсальные методы произвольной стилизации снимают это ограничение, но часто теряют характерные структурные элементы стиля или страдают от артефактов и нестабильного обучения.

В данной работе рассматривается мультистилевая схема рендеринга, где стилевое изображение кодируется в компактный эмбеддинг и внедряется в генератор различными механизмами: прямой конкатенацией к признаковым картам, 1{\texttimes}1-инъекцией (learnable projection), FiLM-модуляцией, а также через патчевое сопоставление признаков. Такой дизайн объединяет скорость feed-forward архитектур с гибкостью произвольной стилизации.

Наш вклад — структурированное сравнение способов внедрения стилевого эмбеддинга, практические рекомендации по устойчивому обучению кодировщика стиля и анализ причин деградации качества (усреднение палитры и мазков, артефакты от локальных сопоставлений). Мы также обсуждаем улучшения: усиление тренировки кодировщика контрастными целями, раздельное внедрение палитры и текстур и альтернативные функции потерь для сохранения локальных деталей.

\medskip
\section{Обзор литературы.}
Базовый подход \citet{gatys2015} формулирует NST как оптимизацию изображения под перцептивные потери на активациях VGG: контент фиксируется через промежуточные признаки, стиль — через матрицы Грама корреляций признаков. Это даёт качественную стилизацию, но непрактично для интерактивных сценариев.

Переход к генераторам реального времени осуществлён в \citet{johnson2016}: сеть-преобразователь с residual-блоками обучается по перцептивным лоссам и стилизует за один прямой проход. Существенную роль в стабильности генерации сыграла Instance Normalization \citep{ulyanov2016instnorm}. Однако такие модели обычно требуют отдельного обучения под каждый стиль.

Поддержка произвольных стилей достигается статистическими методами выравнивания признаков: AdaIN выравнивает среднее и дисперсию каналов \citep{huang2017adain}, WCT применяет whitening\&coloring трансформы \citep{li2017wct}. Работа \citet{ghiasi2017magentanet} использует стиль-кодировщик для предсказания параметров нормализации слоёв генератора, внедряя стиль непосредственно в механизм нормализации. Эти подходы быстры и гибки, но при сильной вариативности стилей могут терять локальные текстуры и характерные мазки.

Локально-структурные (патчевые) методы сопоставляют фрагменты признаков контента и стиля \citep{chen2016fastpatch} или комбинируют CNN с MRF \citep{li2016cnnmrf}, что помогает передавать текстуры и повторяющиеся мотивы, но повышает требования к памяти и склонно к артефактам. Современные модели внимания усиливают согласование стиля и контента (SANet \citep{li2019sanet}, AdaAttN \citep{he2019adaattn}), а трансформерные решения (StyTR$^2$ \citep{xia2022stytr}) улучшают глобальные зависимости ценой усложнения архитектуры и обучения.

Наконец, условные модули (например, FiLM \citep{perez2018film}) линейно модулируют признаки генератора параметрами, зависящими от условия (стиля), и предоставляют простой общий механизм внедрения стилевой информации; их практическая эффективность чувствительна к качеству стилевого эмбеддинга и балансу лоссов.

\section{Headings: first level}
\label{sec:headings}


\subsection{Headings: second level}


\subsubsection{Headings: third level}


\paragraph{Paragraph}


\section{Examples of citations, figures, tables, references}
\label{sec:others}

\subsection{Citations}


\subsection{Figures}


\subsection{Tables}


\subsection{Lists}


\bibliographystyle{unsrtnat}
\bibliography{references}

\end{document}
